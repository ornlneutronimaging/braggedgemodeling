\documentclass{article}
\usepackage[utf8]{inputenc}
\usepackage{authblk}
\usepackage{graphicx}


\title{bem: Modeling for Neutron Bragg-Edge Imaging}
\author[1]{Jiao Y. Y. Lin \thanks{Corresponding author. email: linjiao@ornl.gov}}
\author[1]{Gian Song}
\affil[1]{Neutron Scattering Division, Oak Ridge National Lab}

\date{August 2018}

\begin{document}

\maketitle

\section{Summary}\label{summary}
Due to its zero net charge, neutron is a unique probe of materials.
Low neutron absorption and scattering cross sections by most nuclei make it
suitable for studying bulk samples.
Unlike X-ray scattering, neutron form factors are not monotonically dependent of atomic numbers;
the fact that the neutron scattering cross section of hydrogen is large makes neutron
a useful tool in biology.
In the past half century, Neutron imaging has seen growing applications
in various scientific fields
including physics, engineering sciences, biology, and archaeology
\cite{strobl2009}.

With energy-resolved neutron imaging techniques,
Neutron Bragg-edge imaging has recently found applications for materials science in phase mapping,
stress/strain mapping, and texture analysis
\cite{lehmann2010, sato2017}.
To model Bragg-edge neutron imaging data, it is necessary to calculate
the total neutron cross section of a sample.
This open-source python package
provides easy-to-use functions to calculate coherent elastic (diffraction),
incoherent elastic, coherent inelastic, and incoherent inelastic scattering
cross sections, as well as absorption cross sections
based on approximations and formulas in \cite{vogel2000thesis}.
Also implemented are algorithms that take into account 
the March-Dollase texture model, and the Jorgensen peak profile
\cite{vogel2000thesis}.

\section{Notice of Copyright}\label{notice-of-copyright}

This manuscript has been authored by UT-Battelle, LLC under Contract No.
DE-AC05-00OR22725 with the U.S. Department of Energy. The United States
Government retains and the publisher, by accepting the article for
publication, acknowledges that the United States Government retains a
non-exclusive, paid-up, irrevocable, worldwide license to publish or
reproduce the published form of this manuscript, or allow others to do
so, for United States Government purposes. The Department of Energy will
provide public access to these results of federally sponsored research
in accordance with the DOE Public Access Plan
(http://energy.gov/downloads/doe-public-access-plan).

\section{Acknowledgements}\label{acknowledgements}
This work is sponsored by the Laboratory Directed Research and
Development Program of Oak Ridge National Laboratory, managed by
UT-Battelle LLC, for DOE. Part of this research is supported by the U.S.
Department of Energy, Office of Science, Office of Basic Energy
Sciences, User Facilities under contract number DE-AC05-00OR22725.

\bibliographystyle{unsrt}
\bibliography{paper}

\end{document}
